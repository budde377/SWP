% Dokumentklassen sættes til memoir.
% Manual: http://ctan.org/tex-archive/macros/latex/contrib/memoir/memman.pdf
\documentclass[a4paper,10pt]{article}
\usepackage[a4paper]{geometry}
% Danske udtryk (fx figur og tabel) samt dansk orddeling og fonte med

% danske tegn. Hvis LaTeX brokker sig over æ, ø og å skal du udskifte
% "utf8" med "latin1" eller "applemac". 
\usepackage[utf8]{inputenc}
\usepackage[danish]{babel}
\usepackage[T1]{fontenc}
 
% Matematisk udtryk, fede symboler, theoremer og fancy ting (fx kædebrøker)
\usepackage{amsmath,amssymb}
\usepackage{bm}
\usepackage{amsthm}
\usepackage{tikz}
%\usepackage{mathtools}
% Kodelisting. Husk at læse manualen hvis du vil lave fancy ting.
% Manual: http://mirror.ctan.org/macros/latex/contrib/listings/listings.pdf
\usepackage{listings}
\usepackage{verbatim}
 
% Fancy ting med enheder og datatabeller. Læs manualen til pakken
% Manual: http://www.ctan.org/tex-archive/macros/latex/contrib/siunitx/siunitx.pdf
%\usepackage{siunitx}
% Indsættelse af grafik.
\usepackage{graphicx}
%\usepackage{bussproofs}

% Reaktionsskemaer. Læs manualen for at se eksempler.
% Manual: http://www.ctan.org/tex-archive/macros/latex/contrib/mhchem/mhchem.pdf
%\usepackage[version=3]{mhchem}
\author{Lau Skorstengaard, 20103173 \\Christian Budde Christensen, 20103616}
\title{Serverbaseret Webprogrammering\\1. Aflevering}

\begin{document}
\maketitle
\section*{Opgave 1}
Sammenlignes JSF og Struts ift. hvordan de implementerer MVC-patterenet har JSF en mere direkte adskillelse mellem modellen, viewet og controlleren. Her består modellen af beans, mens viewet består af \texttt{jsf} sider. Controlleren gemt væk i FacesServlet'en, og viewet bestemmer mere direkte hvordan modellen skal opdateres, ved at knytte værdier i modellen, til dele af viewet, og bestemme hvilke metoder der skal kaldes når en formular ``submittes'', på en meget deklarativ måde. Controlleren er på den måde blevet reduceret til en \texttt{faces-config.xml} fil og et requests vej gennem serveren er ikke klar for implementøren af applikationen. En side-effect af denne abstraktion ses på klientens URL, der ikke altid giver mening ift. hvilken side man er på, da en form sender post-request til sig selv.

I Stuts er adskildelsen mellem view og controller mere flydende. Her sørger controlleren i højere grad for at bygge viewet vha. Actions. Denne action-baseret tilgang gør at håndteringen af et request bliver langt mere transparent. Når et request bliver modtaget af serveren, kaldes den rette action. Den konstruerer så et response, som sendes direkte tilbage til klienten.

Ift. code-maintainability, så er gør abstraktionsniveauet i JSF det meget let at fx ændre i modellen og repræsentere ændringerne i viewet. Tilføjes eksempelvis et felt til modellen, kan man repræsentere feltet i viewet ved blot at ændre i den tilhørende XHTML-fil. Er der, som oftest, nogle bestemte krav til værdiers format, tilføjer input kravene ét sted, navnligt i XHTML-filen/erne. Denne feature skulle eftersigende også gøre implementation af AJAX meget let. 

En ulempe ved JSFs abstraktion er at den kan virke som en begrænsning for udvikleren. Eksempelvis har udvikleren begrænset kontrol over hvordan \texttt{jsf/html} forms skal oversættes til html og hvordan AJAX skal håndteres. Derudover er det ikke åbenlyst hvilken tilstand siden har ud fra URL og requests, da meget af sidens tilstand er gemt væk i serverens interne repræsentation af session'en. Dette kan være en hindring ønsker man eksempelvis at lave websider der indeholder information som let skal kunne tilgåes og deles. Her vil man ikke bare kunne dele en URL og så regne med at modtageren åbner den samme side som man selv er på. Stuts, værende et Action framework, passer langt bedre til denne slags sider, med mange brugere. Ønsker man istedet at lave en side med færre brugere og stor fleksibilitet ift. redigering af sidens indhold, så vil et component framework, som JSF er, være at anbefale.

\section*{Opgave 2}

Vi har sørget for at brugeren ikke kan tilgå ``results'' og ``ask'' siderne på følgende måde
\begin{description}
\item[JSF] Vi valgte at implementere dette i de respektive XHTML filer. Dette strider en smule imod tankegangen bag JSF, da det virker mere imparativt end deklarativt, og ideelt set skulle være en del af den bortabstraherede controller. Dette kunne også være implementeret vha. \texttt{PhasesListener}'s, for på den måde at følge tankegangen. 
\item[Struts] Det falde os naturligt at implementere dette som en del af Struts actions. Disse returnerer således \texttt{"FATAL"} hvis spørgsmålet ikke er initialiseret. 
\end{description}
\end{document}

