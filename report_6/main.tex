% Dokumentklassen sættes til memoir.
% Manual: http://ctan.org/tex-archive/macros/latex/contrib/memoir/memman.pdf
\documentclass[a4paper,10pt]{article}
\usepackage[a4paper]{geometry}
% Danske udtryk (fx figur og tabel) samt dansk orddeling og fonte med

% danske tegn. Hvis LaTeX brokker sig over æ, ø og å skal du udskifte
% "utf8" med "latin1" eller "applemac". 
\usepackage[utf8]{inputenc}
\usepackage[danish]{babel}
\usepackage[T1]{fontenc}
 
\usepackage{cite}
\usepackage{natbib}
% Matematisk udtryk, fede symboler, theoremer og fancy ting (fx kædebrøker)
\usepackage{amsmath,amssymb}
\usepackage{bm}
\usepackage{amsthm}
\usepackage{tikz}
%\usepackage{mathtools}
% Kodelisting. Husk at læse manualen hvis du vil lave fancy ting.
% Manual: http://mirror.ctan.org/macros/latex/contrib/listings/listings.pdf
\usepackage{listings}
\usepackage{verbatim}
 
% Fancy ting med enheder og datatabeller. Læs manualen til pakken
% Manual: http://www.ctan.org/tex-archive/macros/latex/contrib/siunitx/siunitx.pdf
%\usepackage{siunitx}
% Indsættelse af grafik.
\usepackage{graphicx}
%\usepackage{bussproofs}

% Reaktionsskemaer. Læs manualen for at se eksempler.
% Manual: http://www.ctan.org/tex-archive/macros/latex/contrib/mhchem/mhchem.pdf
%\usepackage[version=3]{mhchem}
\author{Lau Skorstengaard, 20103173 \\Christian Budde Christensen, 20103616}
\title{Serverbaseret Webprogrammering\\6. Aflevering}

\begin{document}
\maketitle
\section{XDuce, rettelse af buggy.q}

\section{Ligheder og forskelle mellem XSLT, XDuce, Xact, and LINQ for XML}
I de følgende fire afsnit vil vi behandle de fire spørgsmål, som er opstillet i opgaven.
\subsection{Hvordan er XML dokumenter repræsenteret?}
Måden XML dokumenter er repræsenteret på i LINQ og Xact minder lidt om hinanden, da de i begge er repræsenteret som objekter af klasser i et hieraki, der afspejler de forskellige dele af et XML dokument. En forskel mellem LINQ og Xact er dog, at XML dokument objekterne i Xact er immutable, hvor de i LINQ er mutable. Eksempelvis kan man lave et Element objekt, hvor man så kan tilføje et Attribut objekt, hvis det element skal indeholde en attribut, eller man kan tilføje et andet Element objekt, hvis elementet skal have et barn. I XDuce bliver et XML dokumenter repræsenteret ved primitive værdier i modsætning til XACT og LINQ, hvor et dokument bliver repræsenteret af objekter. I XSLT bliver XML dokumentet behandlet næsten ligeom XPath behandler XML dokumenter\footnote{Kilde: http://www.w3.org/TR/xslt#data-model}, hvilket betyder, at det bliver behandlet som et træ, hvor alle knuder har en streng repræsentation. At XSLT betragter XML dokumenter som træer kommer selvfølgelig ikke som en overraskelse, da XML dokumenter i sagens natur er træer. Derfor er det helt naturligt, at strukturen for alle repræsentationerne i bund og grund er træer.

\subsection{Hvordan kan XML dokumenter blive manipuleret og/eller genereret?}
I Xact genereres XML dokumenter ud fra en template, hvor man udfylder nogle gaps i XML dokumentet. Når man har udfyldt alle gaps, eller ikke ønsker at tilføje mere, så lukker man de resterende gaps og får Xact til at give XML dokumentet. XML dokumenter kan også ændres i ved, at man indlæser et eksisterende dokument, hvorefter indsætter gaps, hvor man ønsker at ændre noget. Dette giver selvfølgelig et nyt objekt med nogle gaps, som man nu kan udfylde. I LINQ er det lidt anderledes, da man manipulere dokumentet ved at tilføje og fjerne knuder direkte i træet. Knuder findes ved at lave en ``from, where, orderby, select'' query, ligesom man kender det fra SQL. Dette er forskelligt fra Xact, hvor man bruger XPath til at til at navigere i dokumentet. XSLT bruger også XPath til at finde elementer i et XML dokument, der skal ændres. Ændringen sker ved, at der er nogle template rules, som består af en template og et mønster, hvor mønsteret bruger XPath. Når et dokument så skal manipuleres, så skal findes den template rule, der passer ``bedst'' hvorefter templaten bruges. Slutteligt er der XDuce, som overordnet bygger på matching. Ønsker man eksempelvis at ændre på et XML dokument, så indlæses det, hvorefter der bliver lavet en pattermatching som kendt fra Ocaml. Et helt nyt XML dokument laves i XDuce ved at lave den værdi, der svarer til det ønskede XML dokument. XSLT og XDuce arbejder begge med den slags ``find mønster'' og derefter ``brug template'' metode, hvor man begge steder har regler for hvilket mønster der bruges, hvis der er flere der passer. Eksempelvis bruger XDuce reglen, at det første mønster der passer er det, der bliver brugt.

\subsection{Hvordan annotere programmøren typer for XML værdier?}
I XACT kan et XML objekt annoteres med en type det skal indeholde, eller en type, hvor hvert gap i dokumentet også skal udfyldes med noget af en bestemt type. Man kan også i en template angive typer til gaps, så de skal blive udfyldt med en bestemt type. LINQ to XML arbejder på generiske XML træer, hvilket der ikke er typer i. LINQ to XSD, som er baseret på LINQ to XML, har typer, som bliver udledt fra et XSD Schema.\footnote{Kilde: http://lists.w3.org/Archives/Public/xmlschema-dev/2006Nov/0016.html} I XDuce bruges regular exrepssion types til at give en type til alt i et XML dokument. Det at man på forhånd angiver typerne gør, at man kan bruge OCamls stærke type sytem til statisk at hjælpe med at tjekke om ens patterns er dækkende, og om det er lovlige typer. I XSLT er der ikke mulighed for at annotere typer, men med eksempelvis XSLV (den statiske validering af XSLT transformationer, som er beskrevet i en senere opgave), så giver man de forskellige knuder en type og bruger nogle skemaer til at finde ud af, om det er de korrekte typer - dette er lidt i stil med LINQ to XSD. XSLT, XACT og LINQ ligner hinanden i kraft af, at det er i kraft af et Schema, at man overhovedet kan tildele knuder mv. en type. I XDuce bliver der i modsætning til de øvrige explicit angivet hvilke typer de forskellige knuder har.

\subsection{Hvordan fungerer type systemet og hvilke garantier giver det?}
Hvis vi starter med XACT, så kan det garantere, at outputtet er korrekt relativt til et Schema, der er blevet angivet. Dette er en statisk garanti, men den er konservativ, hvilket betyder, at den kan afvise kode, der altid giver gyldigt output. Den kan dog også lave et tjek på runtime, hvilket ikke er konservativt, da man her har det egentlige output til rådighed. Hvis der er blevet annoteret typer i XACT, så beder man implicit også om at få valideret på runtime samt lavet en statisk analyse. LINQ to XML giver umiddelbart ingen garantier om det XML dokument, der bliver genereret, da der ikke er nogen typer at validere mod. Dog gætter vi på (vi har ikke fundet en kilde til det) at LINQ er i stand til at tjekke et XML dokument relativt til et Schema. XDuce laver en statisk type analyse baseret på de regular expression types, der er angivet. XSLT giver som standard ingen garantier, men kombineres det med den validering af XSLT transformationer, der er angivet senere, så garanteres, at hvis man har et gyldigt XML dokument relativt til et skema, så vil XSLT transformationen give et gyldigt XML dokument relativt til et andet skema. Dette er en statisk, konservativ analyse, hvilket betyder, at den bliver givet når der kompileres, men at der kan forekomme falske positiver, ligesom der kan ved XACT.

\section{Statisk validering af XSLT transformationer}


\end{document}

